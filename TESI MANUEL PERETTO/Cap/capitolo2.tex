\chapter{Il Gradiente di sottosterzo}
\label{cha:cap2}
Il comportamento in curva del veicolo è un fattore critico nel determinare la
sicurezza e le performance di handling dello stesso.
Una classica caratteristica del comportamento in curva è il concetto di
sottosterzo, sovrasterzo o sterzo neutro. Questo concetto è basato dall’interazione 
tra la velocità e la curvatura durante una curva steady-state.
Se all’aumentare della velocità del veicolo cresce la curvatura percorsa allora 
siamo in condizione di sottosterzo.
Quando la curvatura rimane invariata all’aumentare della velocità allora siamo in
condizione di sterzo neutro.
Il gradiente di sottosterzo è una misura che fornisce una misurazione quantitativa 
del comportamento del veicolo (sottosterzante, sovrasterzante, neutro).
Il concetto del gradiente di sottosterzo è stato introdotto per la prima volta da 
Olley nel 1946.
-Olley definì la linea di sterzo neutro come il punto dove qualsiasi forza esterna applicata non produce alcun momento di imbardata addizionale durante una curva. Il gradiente di sottosterzo sarà la distanza tra il centro di massa e la neutral steer line.
Sebbene non siano state ricavate espressioni matematiche questo concetto coinvolge le derivate degli angoli di deriva dei pneumatici, che vengono successivamente ampiamente utilizzati come gradiente dell’angolo di deriva del pneumatico.

-La sterzata di Ackerman è stata proposta nel 1965, in questo caso il gradiente di sottosterzo è definito come la differenza tra l’ackerman steer angle gradient e lo steer angle gradient.

-Pacejka nel 1973 esaminò attentamente il concetto di gradiente di sottosterzo, il comportamento in curva di un veicolo convenzionale è stato esaminato con l’ipotesi di un minimo slip longitudinale del pneumatico e un piccolo angolo di sterzo.



\section{Formulazioni classiche del gradiente di sottosterzo}
Le tre formulazioni maggiormente utilizzate sono:  KPacejka , KSAE , KGuiggiani.
La fonte di questo argomento è ..., mentre ulteriori approfondimenti si trovano in 
\subsection{Formulazione Pacejka}

\subsection{Formulazione Sae}

\subsection{Formulazione Guiggiani}

\subsection{Confronto e limiti delle formulazioni classiche}
Non sono ben definiti nei punti di flesso ecc ecc

\section{Formulazioni alternative del gradiente di sottosterzo}
aggiungere formulazione di Bucchi.



\section{Metodi per la misura del sottosterzo}
\subsection{ISO4138-2021}

Come si.......................
