\chapter{Effetti del modello matematico nella simulazione}
\label{cha:cap4}


In questo capitolo verranno descritti i risultati delle 
simulazioni matlab, evidenziando le differenze che si verificano
sulla base del modello matematico utilizzato per la simulazione.

\section{Alfa Romeo Giulia}
Le simulazioni sono state eseguite utilizzando i dati della 
Alfa Romeo Giulia 2.2 Turbodiesel 150 cv.
I dati sono stati presi da un modello VI-grade presente nei 
database unipd in modo tale da poterne confrontare i risultati.

\subsection{Parametri geometrici}
\begin{table}[H]
    \centering
    \caption{Main vehicle parameters}
    \vspace{0.5em}
    \begin{tabular}{ccc}
    \hline
        Symbol & Name and unit & Value\\
        \hline
        m & Mass [kg] & 1449\\
        J & Moment of inertia of yaw motion [kg $m^2$] & 2129 \\
        l & Wheelbase [m] & 2.82\\
        a & Front semi-wheelbase [m] & 1.315\\
        b & Rear semi-wheelbase [m] & 1.505\\
        $\tau$ & Transmission ratio & 11.8\\
        $W_f$ & Front wheel track [m] & 1.557\\
        $W_r$ & Rear wheel track [m] & 1.625\\
        h & Center of mass height [m] & 0.592\\
        $d_f$ & Front roll center height [m] & 0.041\\
        $d_r$ & Rear roll center height [m] & 0.086\\
    \hline    
    \end{tabular}
    \label{tab:tabella parametri Giulia}
\end{table}

\subsection{Rigidezze}
\begin{table}[H]
    \centering
    \caption{Roll stiffness}
    \vspace{0.5em}
    \begin{tabular}{ccc}
    \hline
        Name & Value & Unit \\
        \hline
        Front suspension stiffness & $20.8*10^3$ & [N/m]\\
        Front rollbar stiffness & $4.168*10^4$ & [Nm/rad]\\
        Rear suspension stiffness & $28.8*10^3$ & [N/m]\\
        Rear rollbar stiffness & $1.1709*10^4$ & [Nm/rad]\\
        Roll stiffness ratio & 0.6468 & [-]\\
    \hline    
    \end{tabular}
    \label{tab:tabella rigidezze}
\end{table}

\subsection{Pneumatici}
Per le simulazioni dei file.tir forniti da Pirelli riguardanti dei pneumatici ----- e -----
e dei pneumatici Toyo del tipo...

\section{Confronti}
I risultati mostrano che Utilizzando il modello single track
indipendentemente dalla prova utilizzata i risultati sul 
gradiente di sottosterzo sono analoghi.
Aumentando la complessità del modello si riscontrano delle differenze
