\emph{\emph{\emph{}}}\appendix

%questa è la parte dell'appendice, sono state inserite due appendici come esempio
%E' sufficiente procedere come al solito, andando ad aggiungere i capitoli e le relative sezioni, etc. La differenza è che ogni capitolo avrà, invece di un numero, una lettera, cosicchè le sezioni diveranno A.1, A.2, e B.1, B.2

\chapter{Glossario}
In questa sezione sono presenti il glossario dei termini usati e un elenco di chiavi di ricerca (\emph{keyword}), con relativa traduzione italiana, utili per il reperimento degli argomenti sui motori di ricerca internazionali.

\section{\label{sec:Glossario}Glossario di riferimento}
\begin{itemize}
\item \textbf{ABDUZIONE} - \emph{abduction} \\
Movimento di allontanamento del braccio dal tronco.
\item \textbf{ADDUZIONE} - \emph{adduction} \\
Movimento di avvicinamento del braccio dal tronco.
\item \textbf{APPARATO CAPSULO-LEGAMENTOSO} \\
Sono strutture deputate a mantenere unite le estremità di due ossa contigue, e
allo stesso tempo permettere il movimento di una rispetto all'altra (articolarità).
\item \textbf{AFASIA} \\
Perdita della capacità di comunicare oralmente, per segno o per iscritto, oppure
incapacità di comprendere tali forme di comunicazione, perdita della capacità di
usare il linguaggio.
\item \textbf{AGRAFIA} \\
Incapacità di esprimere per iscritto il proprio pensiero.
\item ...
\end{itemize}

\section{\label{sec:Chiavi}Chiavi di ricerca}
\begin{table}[h]
	\centering
		\begin{tabular}{c|c}
		\textbf{Inglese} & \textbf{Italiano} \\
			\hline
			post stroke rehabilitation & riabilitazione post ictus \\
			upper limb & arto superiore \\
			wire-driven robot & robot a cavi \\
			degrees of freedom & gradi di libertà \\
			robot-aided rehabilitation & riabilitazione assistita da robot \\
			subacute phase & fase acuta \\
			chronic-phase & fase cronica \\
			\end{tabular}	
	\label{tab:CorrispondenzeItalianoInglese}
\end{table}
 
 
\chapter{\label{sec:CoordOmogenee}Le coordinate omogenee}
L'uso delle coordinate omogenee permette
di rappresentare in modo alternativo funzioni razionali, quali
quelle viste per le curve di B\'ezier razionali (eq.
%\ref{}) e le Nurbs (eq. \ref{}).
L'idea \`e di rappresentare una curva razionale in uno spazio a
\emph{n} dimensioni, come una curva polinomiale (non razionale) in
uno spazio a $n+1$ dimensioni. Un punto in uno spazio 3D, scritto
come $\textbf{P}=(x,y,z)$, diventa $\textbf{P}^{w}=(wx,wy,wz,w)$,
con $w\neq 0$. La trasformazione che permette di passare dalla
coordinate omogenee a quelle ``normali'' \`e:
\begin{displaymath}
\textbf{P}=H\{\textbf{P}^{w}\}=H\{(X,Y,Z,W)\}=H\Big\{ \Big( \frac{X}{W},\frac{Y}{W},\frac{Z}{W} \Big) \Big\}
\end{displaymath}
Ora, dato un insieme di punti di controllo $\{\textbf{P}_{i}\}$ e
di pesi $w_{i}$ \`e possibile costruire dei punti di controllo
pesati $\textbf{P}_{i}^{w}=(w_{i}x,w_{i}y,w_{i}z,w_{i})$, e
definire curve di B\'ezier non-razionali in uno spazio a quattro
dimensioni
\begin{displaymath}
\textbf{C}^{w}(u)=\sum_{i=0}^{n}B_{i,p}(u)\textbf{P}^{w}_{i}
\end{displaymath}
Poi, applicando la trasformazione vista sopra, si arriva alla forma razionale vista in eq. %\ref{eq:bezierrazionali}; come esempio vediamo la prima coordinata
\begin{displaymath}
x(u)=\frac{X(u)}{W(u)}=\frac{\displaystyle\sum_{i=0}^{n}B_{i,p}(u)w_{i}x_{i}}
{\displaystyle\sum_{i=0}^{n}B_{i,n}(u)w_{i}}
\end{displaymath}
In questo modo possiamo definire anche superfici razionali come quella gi\`a vista a pag.%\pageref{eq:superficiebezierconw} e che qui viene riportata
\begin{displaymath}
\textbf{S}^{w}(u,v)=\sum_{i=0}^{n}\sum_{j=0}^{m}B_{i,p}(u)B_{j,q}(v)\textbf{P}^{w}_{i,j}
\end{displaymath}
da cui si ottiene, portandola allo spazio a 3 dimensioni:
\begin{displaymath}
\textbf{S}(u,v)=H\{\textbf{S}^{w}(u,v)\}=
\frac{\sum_{i=0}^{n}\sum_{j=0}^{m}B_{i,p}(u)B_{j,q}(v)w_{i,j}\textbf{P}_{i,j}}
{\sum_{i=0}^{n}\sum_{j=0}^{m}B_{i,p}(u)B_{j,q}(v)w_{i,j}}
\end{displaymath}

