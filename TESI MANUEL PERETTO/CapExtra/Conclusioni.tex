\chapter*{Conclusioni}
\markboth{CONCLUSIONI}{CONCLUSIONI}
\thispagestyle{empty}
\addcontentsline{toc}{chapter}{Conclusioni}


In questo lavoro di tesi sono stati sviluppati diversi programmi,
sia in ambiente MatLAB che in VisualC++, allo scopo di testare
l'interfaccia aptica Piroga5, utilizzandola
nella simulazione di un ambiente virtuale.

Tale ambiente virtuale è stato creato con l'utilizzo di superfici
NURBS, gestite da routine apposite, sviluppate in lavori precedenti, il cui codice è stato riscritto utilizzando la struttura a classi, e trattando quindi
 ogni entità NURBS come un oggetto, con le sue variabili ed i suoi metodi.

\`E stato \textit{sviluppato} un programma, che gestisce
l'ambiente, restituendo i valori delle forze di reazione
conseguenti agli urti con gli oggetti virtuali, e grazie a dei metodi innovativi nel calcolo del punto di minima
distanza tra la punta dell'end-effector e la superficie
dell'oggetto virtuale, la velocità di elaborazione è
stata incrementata molto, consentendo di superare il KHz nelle frequenze di lavoro. 
Ciò ha permesso di
creare ambienti non più limitati ad una singola superficie: con
due oggetti la frequenza di lavoro si aggira sui 1100Hz. Inoltre sono state implementate 
anche superfici con caratteristiche di
rigidezza variabile da punto a punto e in base anche allo spessore di
penetrazione.

Uno dei possibili sviluppi di questo lavoro è sicuramente rappresentato dalle MetaNurbs, ultimando i metodi e le tecniche già iniziati e purtroppo non completati per mancanza di tempo. Quando esse saranno completamente operative, permetteranno di raggiungere frequenze di lavoro elevatissime consentendo così di gestire molte più superfici contemporaneamente. A tal proposito andranno necessariamente migliorate le routine per l'importazione di NURBS da file IGES.


\clearpage
\thispagestyle{empty}
\cleardoublepage
