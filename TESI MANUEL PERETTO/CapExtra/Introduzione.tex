%Le 2 righe iniziali hanno lo scopo di formattare un capitolo evitando la scritta Capitolo 1, aggiungendolo comunque
%all'indice (toc=table of contents) e definendo le intestazioni di pagina in modo adatto

\chapter*{Introduzione\markboth{\MakeUppercase{Introduzione}}{}}
\addcontentsline{toc}{chapter}{Introduzione}



La tesi che viene descritta nelle pagine seguenti si inserisce nell'ambito delle analisi di stabilità in curva delle autovetture.
\textbf{.....} (.......), 

In particolare, come i modelli matematici utilizzati per 
simulare il comportamento in curva dei veicoli influenzino i 
risultati dello stesso. 

La tesi ha riguardato lo studio delle classiche formulazioni 
del gradiente di sottosterzo e lo sviluppo di alcuni modelli
matematici su ambiente matlab che permettessero di simulare le 
manovre caratteristiche specificate in normativa tramite le
quali analizzare il comportamento in curva dei veicoli.\\


Il capitolo \ref{cha:cap1} è introduttivo al lavoro svolto e 
consiste in un riepilogo specifico sui fondamenti di teoria 
necessari a comprendere la fisica utilizzata nei modelli. 
A partire dalla teoria sui pneumatici per poi passare alla descrizione dell'influenza dei parametri geometrici del veicolo e dei vari componenti sulla risposta dello stesso rispetto alle diverse sollecitazioni impartite del conducente. 
\textbf{...}
.\\

Le comuni autovetture stradali sono progettate in modo da garantire la maggiore 
stabilità possibile durante una curva, nel capitolo \ref{cha:cap2} vengono illustrate le più importanti formulazioni matematiche utilizzate per descrivere e stimare il comportamento delle
stesse durante manovre di test quasi statiche.\\

Attualmente esistono numerose formulazioni create da differenti autori ed ognuna 
presenta pregi e difetti. Lo scopo di questo capitolo è di analizzarle e verificarne i limiti, tramite simulazioni matlab. 
Successivamente si è vagliata l'attendibilità e l'efficacia di nuove formulazioni proposte ancora in fase di revisione.\\
La seconda parte della tesi riguarda la realizzazione di un modello di veicolo che possa replicare in modo accurato la risposta di un veicolo reale. 
Utilizzando le geometrie di 
un alfa romeo Giulia si è potuto evidenziare come, partendo da un modello semplice ed arrivando
ad un modello via via più complesso si possano ottenere 
risultati diversi ma più simili alle condizioni reali di funzionamento.\\
Infine i risultati ottenuti sono stati confrontati con simulazioni eseguite tramite software di simulazione multibody, per valutarne e comprenderne le eventuali differenze.




\clearpage
\thispagestyle{empty}
\cleardoublepage
